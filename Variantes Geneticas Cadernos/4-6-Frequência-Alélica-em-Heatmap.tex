% Options for packages loaded elsewhere
\PassOptionsToPackage{unicode}{hyperref}
\PassOptionsToPackage{hyphens}{url}
%
\documentclass[
]{article}
\usepackage{lmodern}
\usepackage{amssymb,amsmath}
\usepackage{ifxetex,ifluatex}
\ifnum 0\ifxetex 1\fi\ifluatex 1\fi=0 % if pdftex
  \usepackage[T1]{fontenc}
  \usepackage[utf8]{inputenc}
  \usepackage{textcomp} % provide euro and other symbols
\else % if luatex or xetex
  \usepackage{unicode-math}
  \defaultfontfeatures{Scale=MatchLowercase}
  \defaultfontfeatures[\rmfamily]{Ligatures=TeX,Scale=1}
\fi
% Use upquote if available, for straight quotes in verbatim environments
\IfFileExists{upquote.sty}{\usepackage{upquote}}{}
\IfFileExists{microtype.sty}{% use microtype if available
  \usepackage[]{microtype}
  \UseMicrotypeSet[protrusion]{basicmath} % disable protrusion for tt fonts
}{}
\makeatletter
\@ifundefined{KOMAClassName}{% if non-KOMA class
  \IfFileExists{parskip.sty}{%
    \usepackage{parskip}
  }{% else
    \setlength{\parindent}{0pt}
    \setlength{\parskip}{6pt plus 2pt minus 1pt}}
}{% if KOMA class
  \KOMAoptions{parskip=half}}
\makeatother
\usepackage{xcolor}
\IfFileExists{xurl.sty}{\usepackage{xurl}}{} % add URL line breaks if available
\IfFileExists{bookmark.sty}{\usepackage{bookmark}}{\usepackage{hyperref}}
\hypersetup{
  pdftitle={Caderno Computacional 4/6: Mesclagem de Arquivos VCF e Frequência Alélica de Polimorfismos de SARS-CoV-2},
  pdfauthor={Gepoliano Chaves, Ph. D.},
  hidelinks,
  pdfcreator={LaTeX via pandoc}}
\urlstyle{same} % disable monospaced font for URLs
\usepackage[margin=1in]{geometry}
\usepackage{color}
\usepackage{fancyvrb}
\newcommand{\VerbBar}{|}
\newcommand{\VERB}{\Verb[commandchars=\\\{\}]}
\DefineVerbatimEnvironment{Highlighting}{Verbatim}{commandchars=\\\{\}}
% Add ',fontsize=\small' for more characters per line
\usepackage{framed}
\definecolor{shadecolor}{RGB}{248,248,248}
\newenvironment{Shaded}{\begin{snugshade}}{\end{snugshade}}
\newcommand{\AlertTok}[1]{\textcolor[rgb]{0.94,0.16,0.16}{#1}}
\newcommand{\AnnotationTok}[1]{\textcolor[rgb]{0.56,0.35,0.01}{\textbf{\textit{#1}}}}
\newcommand{\AttributeTok}[1]{\textcolor[rgb]{0.77,0.63,0.00}{#1}}
\newcommand{\BaseNTok}[1]{\textcolor[rgb]{0.00,0.00,0.81}{#1}}
\newcommand{\BuiltInTok}[1]{#1}
\newcommand{\CharTok}[1]{\textcolor[rgb]{0.31,0.60,0.02}{#1}}
\newcommand{\CommentTok}[1]{\textcolor[rgb]{0.56,0.35,0.01}{\textit{#1}}}
\newcommand{\CommentVarTok}[1]{\textcolor[rgb]{0.56,0.35,0.01}{\textbf{\textit{#1}}}}
\newcommand{\ConstantTok}[1]{\textcolor[rgb]{0.00,0.00,0.00}{#1}}
\newcommand{\ControlFlowTok}[1]{\textcolor[rgb]{0.13,0.29,0.53}{\textbf{#1}}}
\newcommand{\DataTypeTok}[1]{\textcolor[rgb]{0.13,0.29,0.53}{#1}}
\newcommand{\DecValTok}[1]{\textcolor[rgb]{0.00,0.00,0.81}{#1}}
\newcommand{\DocumentationTok}[1]{\textcolor[rgb]{0.56,0.35,0.01}{\textbf{\textit{#1}}}}
\newcommand{\ErrorTok}[1]{\textcolor[rgb]{0.64,0.00,0.00}{\textbf{#1}}}
\newcommand{\ExtensionTok}[1]{#1}
\newcommand{\FloatTok}[1]{\textcolor[rgb]{0.00,0.00,0.81}{#1}}
\newcommand{\FunctionTok}[1]{\textcolor[rgb]{0.00,0.00,0.00}{#1}}
\newcommand{\ImportTok}[1]{#1}
\newcommand{\InformationTok}[1]{\textcolor[rgb]{0.56,0.35,0.01}{\textbf{\textit{#1}}}}
\newcommand{\KeywordTok}[1]{\textcolor[rgb]{0.13,0.29,0.53}{\textbf{#1}}}
\newcommand{\NormalTok}[1]{#1}
\newcommand{\OperatorTok}[1]{\textcolor[rgb]{0.81,0.36,0.00}{\textbf{#1}}}
\newcommand{\OtherTok}[1]{\textcolor[rgb]{0.56,0.35,0.01}{#1}}
\newcommand{\PreprocessorTok}[1]{\textcolor[rgb]{0.56,0.35,0.01}{\textit{#1}}}
\newcommand{\RegionMarkerTok}[1]{#1}
\newcommand{\SpecialCharTok}[1]{\textcolor[rgb]{0.00,0.00,0.00}{#1}}
\newcommand{\SpecialStringTok}[1]{\textcolor[rgb]{0.31,0.60,0.02}{#1}}
\newcommand{\StringTok}[1]{\textcolor[rgb]{0.31,0.60,0.02}{#1}}
\newcommand{\VariableTok}[1]{\textcolor[rgb]{0.00,0.00,0.00}{#1}}
\newcommand{\VerbatimStringTok}[1]{\textcolor[rgb]{0.31,0.60,0.02}{#1}}
\newcommand{\WarningTok}[1]{\textcolor[rgb]{0.56,0.35,0.01}{\textbf{\textit{#1}}}}
\usepackage{graphicx,grffile}
\makeatletter
\def\maxwidth{\ifdim\Gin@nat@width>\linewidth\linewidth\else\Gin@nat@width\fi}
\def\maxheight{\ifdim\Gin@nat@height>\textheight\textheight\else\Gin@nat@height\fi}
\makeatother
% Scale images if necessary, so that they will not overflow the page
% margins by default, and it is still possible to overwrite the defaults
% using explicit options in \includegraphics[width, height, ...]{}
\setkeys{Gin}{width=\maxwidth,height=\maxheight,keepaspectratio}
% Set default figure placement to htbp
\makeatletter
\def\fps@figure{htbp}
\makeatother
\setlength{\emergencystretch}{3em} % prevent overfull lines
\providecommand{\tightlist}{%
  \setlength{\itemsep}{0pt}\setlength{\parskip}{0pt}}
\setcounter{secnumdepth}{-\maxdimen} % remove section numbering

\title{Caderno Computacional 4/6: Mesclagem de Arquivos VCF e Frequência
Alélica de Polimorfismos de SARS-CoV-2}
\author{Gepoliano Chaves, Ph. D.}
\date{Setembro, 2020}

\begin{document}
\maketitle

{
\setcounter{tocdepth}{5}
\tableofcontents
}
\hypertarget{introduuxe7uxe3o}{%
\section{Introdução}\label{introduuxe7uxe3o}}

O Caderno 2 destinou-se a introduzir o conceito de Associação Biológica,
através da associação estatística entre variantes localizadas nos genes
de sortilinas e a Doença de Huntington. No Caderno 3, estudamos o
formato do arquivo de sequência biológica do tipo FASTA. Também no
Caderno 3, utilizamos um \emph{script} escrito em Bash para identificar
variantes de SARS-CoV-2 associadas a uma região geográfica específica.

Caso ainda não tenha conseguido executar comandos em Bash, ou mesmo
instalar linux em sua máquina que usa o sistema operacional Windows, não
se preocupe pois o contato inicial com pipelines computacionais pode
exigir habilidades que nem todos possuimos de imediato. Portanto, a
continuação do estudo nesta área requer persistência e dedicação na
aprendizagem da sintaxe adequada a cada linguagem de programação e
conhecimento dos sistemas operacionais das máquinas utilizadas por você.

No presente Caderno, desejamos mesclar arquivos VCF provenientes da
pipeline de identificação de variantes (\emph{Variant Call}) genômicas
de SARS-CoV-2, produzidos no Caderno 2. A identificação dos
polimorfismos de SARS-CoV-2 foi feita no Caderno Computacional 2/6,
utilizando-se os arquivos FASTA provenientes da plataforma de dados
GISAID. A necessidade de mesclar os arquivos VCF de cada amostra de
SARS-CoV-2 vem de podermos continuar nosso estudo usando os arquivos VCF
de duas formas. Na primeira maneira, comparamos as variantes que
identificamos de modo a considerar nossas próprias hipóteses, o que
poderia acontecer caso fôssemos um grupo de virologia tradicional, que
trabalhasse com SARS-CoV-2 há algum tempo. Neste caso, poderíamos fazer
análises que não necessitassem comparação com dados de grupos mais
consolidados nesta área.

Entretanto, como nosso estudo é exploratório, devemos seguir uma segunda
rota, que inclui a comparação com dados de grupos que publicaram
anteriormente sobre identificação de variantes de SARS-CoV, para
eventualmente, fazermos as comparações científicas necessárias. Desta
forma, pode ser vantajoso utilizarmos uma tabela de variantes de
SARS-CoV-2 publicada por outros pesquisadores. O trabalho de Yin e
colaboradores, publicado em 2020, traz uma tabela com as frequências de
mutações encontradas por este grupo (Yin \emph{et al.}, 2020). Numa
abordagem inicial de nosso estudo, desejamos incluir todas os SNPs
indentificadas por Yin em uma tabela de frequência alélica, para
comparação. Eventualmente, podemos desejar incluir todas as SNPs
identificadas por nossa pipeline, ao invés de incluir apenas as SNPs de
Yin \emph{et al.}.

Durante o programa de Estágio de Verão, na Universidade da Califórnia em
Santa Cruz em 2020, inicialmente construímos esta tabela de frequências
usando o Bash, atraves do comando \emph{grep}. Nesta estratégia, a
posição genômica é identificada diretamente do arquivo \emph{vcf}, que
pode ser o arquivo VCF de cada amostra, ou o arquivo VCF contendo todas
as amostras combinadas, com uma leve preferência pela primeira
abordagem.

Nenhuma destas abordagens, no entanto, parece eficaz. Suponho que uma
abordagem mais eficiente possa ser oferecida por pacotes como SNPhylo,
que parece conseguir plotar a árvore filogenética diretamente dos
arquivos \emph{vcf}s provenientes da pipeline de \emph{Variant Call}.
Para o futuro de nosso estudo, pode ser interessante fazer a analise
filogenética usando o pacote SNPPhylo.

Para uma intodução ao pacote snphylo, bem como visualização do
\emph{workflow} que utiliza, visitar a \emph{home page} do projeto:

\url{http://chibba.pgml.uga.edu/snphylo}

Outro pacote que parece ser bastante útil seria VCF-kit. A \emph{home
page} do pacote VCF-kit pode ser acessada em

\url{https://vcf-kit.readthedocs.io/en/latest/}

Abaixo, o diretório se chama Merged\_Autralia, porque comecei mesclando
os arquivos VCF correspondentes à região da Austrália.

\hypertarget{criar-pasta-para-armazenamento-de-arquivos-vcfs}{%
\section{1) Criar pasta para Armazenamento de arquivos
VCFs}\label{criar-pasta-para-armazenamento-de-arquivos-vcfs}}

Podemos fazer do Rmd notebook uma plataforma de interacao com o nosso
computador. A proxima linha de codigo, cria uma pasta ou diretorio no
Desktop do meu computador (a hierarquia das pastas no seu sistema
computacional determinara a linha exata a ser escrita abaixo). O
estudante precisa descobrir como declarar o seu proprio Desktop na linha
abaixo.

\hypertarget{criar-pasta-para-arquivos-vcf}{%
\subsection{1.1) Criar pasta para arquivos
VCF}\label{criar-pasta-para-arquivos-vcf}}

Na pasta criada, colocaremos todos os arquivos VCF. Este passo eh
necessario para o passo 8 de mesclagem, abaixo. Crie a pasta onde serao
armazenados os arquivos VCF.

\begin{Shaded}
\begin{Highlighting}[]
\FunctionTok{mkdir}\NormalTok{ ~/Desktop/Australia_Merged}
\end{Highlighting}
\end{Shaded}

Agora, precisamos copiar os nomes dos arquivos VCF com base na
localizacao dos arquivos na corrente pasta. Usamos uma ``for loop'',
onde cada pasta i, tera copiado (cp) o arquivo bwa\_aligned\_snps.vcf
modificado para conter o nome da regiao de onde o arquivo eh proveniente
(neste caso, Australia)

\hypertarget{renomear-arquivos-vcf}{%
\subsection{1.2) Renomear arquivos VCF}\label{renomear-arquivos-vcf}}

Renomear dos arquivos VCF eh necessario pois foram todos produzidos como
``bwa\_aligned\_snps.vcf'' pela pipeline GATK.

\begin{Shaded}
\begin{Highlighting}[]
\KeywordTok{for} \ExtensionTok{i}\NormalTok{ in Australia_EPI_ISL_416*}\KeywordTok{;} \KeywordTok{do}
\FunctionTok{cp} \VariableTok{$i}\NormalTok{/bwa_aligned_snps.vcf }\VariableTok{$i}\NormalTok{/}\VariableTok{$\{i:0:24\}}\StringTok{"_bwa_aligned_snps.vcf"}\KeywordTok{;} \KeywordTok{done}
\end{Highlighting}
\end{Shaded}

\hypertarget{uso-de-awk-para-isolar-o-nome-do-arquivo-de-interesse}{%
\subsection{1.3) Uso de ``awk'' para isolar o nome do arquivo de
interesse}\label{uso-de-awk-para-isolar-o-nome-do-arquivo-de-interesse}}

Aqui, usamos a funcao ``echo'' para extrair uma lista contendo os
arquivos VCF que pretendemos analisar. Esta lista sera usada no passo 8
para mesclar os arquivos VCF. Este comando imprime o ``path'' ou caminho
de localizacao do presente diretorio. No entanto, usamos uma barra como
separador para o comando awk, de modo a podermos isolar o nome do
arquivo como desejamos.

\begin{Shaded}
\begin{Highlighting}[]
\KeywordTok{for} \ExtensionTok{i}\NormalTok{ in Australia_EPI_ISL_416*/Australia*}\KeywordTok{;} \KeywordTok{do} \BuiltInTok{echo} \VariableTok{$i}\KeywordTok{;} \KeywordTok{done} \KeywordTok{|} \FunctionTok{awk}\NormalTok{ -F }\StringTok{'/'} \StringTok{'\{print $2\}'}
\end{Highlighting}
\end{Shaded}

\hypertarget{mover-vcfs-para-pasta-criada-inicialmente}{%
\subsection{1.4) Mover VCFs para pasta criada
inicialmente}\label{mover-vcfs-para-pasta-criada-inicialmente}}

Se os arquivos VCF ainda nao foram movidos para a pasta que criamos
inicialmente, mova-os agora.

\begin{Shaded}
\begin{Highlighting}[]
\FunctionTok{mv}\NormalTok{ Australia_EPI*/Australia* Australia_Merged}
\BuiltInTok{cd}\NormalTok{ Australia_Merged}
\end{Highlighting}
\end{Shaded}

\hypertarget{compressuxe3o-de-arquivos}{%
\section{2) Compressão de Arquivos}\label{compressuxe3o-de-arquivos}}

\hypertarget{instalauxe7uxe3o-de-bgzip-e-anaconda-e-compressuxe3o}{%
\subsection{2.1) Instalação de bgzip e Anaconda e
compressão}\label{instalauxe7uxe3o-de-bgzip-e-anaconda-e-compressuxe3o}}

O programa bgzip precisa ser instalado para atender aos requerimentos do
software bcftools. No servidor da UCSC, instalei bgzip com o comando
abaixo. Eesse comando utiliza anaconda, uma ferramenta usada em Ciencia
de Dados.

\begin{Shaded}
\begin{Highlighting}[]
\ExtensionTok{conda}\NormalTok{ install -c bioconda tabix }
\KeywordTok{for} \ExtensionTok{x}\NormalTok{ in }\VariableTok{$(}\FunctionTok{cat}\NormalTok{ Australia_files_list.txt}\VariableTok{)}\KeywordTok{;} \KeywordTok{do} \ExtensionTok{bgzip} \VariableTok{$x}\KeywordTok{;} \KeywordTok{done}
\end{Highlighting}
\end{Shaded}

Os arquivos agora são comprimidos para utilização por bcftools.

\begin{Shaded}
\begin{Highlighting}[]
\KeywordTok{for} \ExtensionTok{x}\NormalTok{ in }\VariableTok{$(}\FunctionTok{cat}\NormalTok{ Australia_files_list.txt}\VariableTok{)}\KeywordTok{;} \KeywordTok{do} \ExtensionTok{bgzip} \VariableTok{$x}\KeywordTok{;} \KeywordTok{done}
\end{Highlighting}
\end{Shaded}

\hypertarget{indexamento}{%
\subsection{2.2) Indexamento}\label{indexamento}}

O indexamento também é requerido por bcftools

\begin{Shaded}
\begin{Highlighting}[]
\KeywordTok{for} \ExtensionTok{x}\NormalTok{ in }\VariableTok{$(}\FunctionTok{cat}\NormalTok{ Australia_files_list.txt}\VariableTok{)}\KeywordTok{;} \KeywordTok{do} \ExtensionTok{tabix} \VariableTok{$x}\StringTok{".gz"}\KeywordTok{;} \KeywordTok{done}
\end{Highlighting}
\end{Shaded}

\hypertarget{copiar-arquivos-vcfs-originais}{%
\subsection{2.3) Copiar arquivos VCFs
originais}\label{copiar-arquivos-vcfs-originais}}

Os arquivos VCF foram usados no passo de indexamento. Assim eh
necessario re-colocar os arquivos na pasta inicial para prosseguir com a
etapa de mesclagem.

\begin{Shaded}
\begin{Highlighting}[]
\KeywordTok{for} \ExtensionTok{i}\NormalTok{ in Australia_EPI_ISL_416*}\KeywordTok{;} 
\KeywordTok{do} \FunctionTok{cp} \VariableTok{$i}\NormalTok{/bwa_aligned_snps.vcf }\VariableTok{$i}\NormalTok{/}\VariableTok{$\{i:0:24\}}\StringTok{"_bwa_aligned_snps.vcf"}\KeywordTok{;} \KeywordTok{done}
\FunctionTok{mv}\NormalTok{ Australia_EPI_ISL_416*/Australia* Australia_Merged}
\end{Highlighting}
\end{Shaded}

\hypertarget{mesclagem-de-arquivos-vcf}{%
\subsection{2.4) Mesclagem de arquivos
VCF}\label{mesclagem-de-arquivos-vcf}}

\begin{Shaded}
\begin{Highlighting}[]
\ExtensionTok{bcftools}\NormalTok{ merge --missing-to-ref --force-samples}

\ExtensionTok{Australia_EPI_ISL_416412_bwa_aligned_snps.vcf.gz}
\ExtensionTok{Australia_EPI_ISL_416413_bwa_aligned_snps.vcf.gz}
\ExtensionTok{Australia_EPI_ISL_416414_bwa_aligned_snps.vcf.gz}
\ExtensionTok{Australia_EPI_ISL_416415_bwa_aligned_snps.vcf.gz}
\ExtensionTok{Australia_EPI_ISL_416514_bwa_aligned_snps.vcf.gz}
\ExtensionTok{Australia_EPI_ISL_416515_bwa_aligned_snps.vcf.gz}
\ExtensionTok{Australia_EPI_ISL_416516_bwa_aligned_snps.vcf.gz}
\ExtensionTok{Australia_EPI_ISL_416517_bwa_aligned_snps.vcf.gz}
\ExtensionTok{Australia_EPI_ISL_416518_bwa_aligned_snps.vcf.gz}

\OperatorTok{>} \ExtensionTok{Australia_Merged.vcf}
\end{Highlighting}
\end{Shaded}

\hypertarget{heatmap-e-agrupamento-hierarquico-hierarchical-clustering}{%
\section{3) Heatmap e Agrupamento Hierarquico (Hierarchical
Clustering)}\label{heatmap-e-agrupamento-hierarquico-hierarchical-clustering}}

Aqui, começamos a observar como os diferentes genotipos do virus se
dividem ao redor do globo, nas diferentes regioes geograficas. O Heatmap
ilustra a frequencia de cada genotipo em diferentes regioes geograficas.

\hypertarget{extrauxe7uxe3o-de-genotipos-descritos-em-yin-2020}{%
\subsection{3.1) Extração de genotipos descritos em Yin
2020}\label{extrauxe7uxe3o-de-genotipos-descritos-em-yin-2020}}

Usamos uma ``for loop'' para extrair cada SNP do arquivo VCF.

\begin{Shaded}
\begin{Highlighting}[]
\KeywordTok{for} \ExtensionTok{x}\NormalTok{ in Australia_EPI_*}\KeywordTok{;} \KeywordTok{do} \FunctionTok{grep}\NormalTok{ SNP }\VariableTok{$x}\NormalTok{/bwa_aligned_snps.vcf}\KeywordTok{;} \KeywordTok{done}
\KeywordTok{for} \ExtensionTok{x}\NormalTok{ in Australia_EPI_*}\KeywordTok{;} \KeywordTok{for} \ExtensionTok{y}\NormalTok{ in list}\KeywordTok{;} \KeywordTok{do} \FunctionTok{grep} \VariableTok{$y}\NormalTok{ vcf_file}\KeywordTok{;}
\end{Highlighting}
\end{Shaded}

Tambem extraimos um SNP de interesse usando o comando grep, ilustrado na
ultima aula e no comando acima.

\begin{Shaded}
\begin{Highlighting}[]
\FunctionTok{grep}\NormalTok{ 12357 ~/COVID_BWA_Variant_Call/Australia_EPI_ISL_416412.fasta/bwa_aligned_snps.vcf}
\end{Highlighting}
\end{Shaded}

Para extrair todas as variantes ao mesmo tempo, podemos usar o seguinte
comando.

\begin{Shaded}
\begin{Highlighting}[]
\FunctionTok{grep}\NormalTok{ 11083 ~/COVID_BWA_Variant_Call/Australia_EPI_ISL*/bwa_aligned_snps.vcf}
\end{Highlighting}
\end{Shaded}

O uso do asterisco, ou estrela, permite o computador entenda que
queremos todos os arquivos que possuem o padrao Australia\_EPI\_ISL, na
pasta \textasciitilde/COVID\_BWA\_Variant\_Call. Este padrao pode ser
observado quando peco ao computador para listar todos os arquivos da
pasta \textasciitilde/COVID\_BWA\_Variant\_Call/, que possuem o padrao
Australia\_EPI\_ISL:

\begin{Shaded}
\begin{Highlighting}[]
\FunctionTok{ls}\NormalTok{ ~/COVID_BWA_Variant_Call/Australia_EPI_ISL*}
\end{Highlighting}
\end{Shaded}

\hypertarget{plotagem-de-heatmap}{%
\subsection{3.2) Plotagem de Heatmap}\label{plotagem-de-heatmap}}

\hypertarget{carregar-dados-de-frequencia-de-polimorfismos}{%
\subsubsection{3.2.1) Carregar dados de frequencia de
polimorfismos}\label{carregar-dados-de-frequencia-de-polimorfismos}}

Vamos inicialmente, visualizar a tabela construida a partir da
genotipagem de SARS-CoV-2 presente nos arquivos VCF. A genotipagem foi
feita por meio da identificacao das SNPs nos diversos arquivos VCF. Uma
vez que consigo contar em quantos arquivos VCF ha determinado SNP de
interesse, consigo fazer uma mensuracao da frequencia desse SNP entre
todos os arquivos VCF que analisei. A visualizacao abaixo eh feita
usando-se o Linux (Bash).

\begin{Shaded}
\begin{Highlighting}[]
\FunctionTok{head}\NormalTok{ ~/Desktop/Gepoliano/SIP2020/Code/temp_file.txt}
\FunctionTok{awk} \StringTok{'\{print $1\}'}\NormalTok{ ~/Desktop/Gepoliano/SIP2020/Code/temp_file.txt}
\end{Highlighting}
\end{Shaded}

A Tabela para Heatmap contém os valores das frequências das SNPs
identificadas em SARS-CoV-2 em diferentes locais do mundo e pode ser
construída usando a extração do genótipo SNP mostrada na etapa 3.

\begin{Shaded}
\begin{Highlighting}[]
\KeywordTok{library}\NormalTok{(}\StringTok{"pheatmap"}\NormalTok{)}
\KeywordTok{library}\NormalTok{(}\StringTok{"RColorBrewer"}\NormalTok{)}
\KeywordTok{setwd}\NormalTok{(}\StringTok{"~/Desktop/Gepoliano/SIP2020/Code"}\NormalTok{)}

\NormalTok{heatmap_table <-}\StringTok{ }\KeywordTok{read.table}\NormalTok{(}\StringTok{"~/Desktop/Gepoliano/SIP2020/Code/temp_file.txt"}\NormalTok{, }\DataTypeTok{row.names =} \DecValTok{1}\NormalTok{, }\DataTypeTok{header =} \OtherTok{TRUE}\NormalTok{, }\DataTypeTok{sep =} \StringTok{"}\CharTok{\textbackslash{}t}\StringTok{"}\NormalTok{)}
\NormalTok{heatmap_table =}\StringTok{ }\KeywordTok{as.matrix}\NormalTok{(heatmap_table)}
\end{Highlighting}
\end{Shaded}

\hypertarget{plotagem-tabela-frequuxeancia-aluxe9lica-sem-normalizauxe7uxe3o}{%
\subsubsection{3.2.2) Plotagem Tabela Frequência Alélica Sem
Normalização}\label{plotagem-tabela-frequuxeancia-aluxe9lica-sem-normalizauxe7uxe3o}}

Note que ao plotar o heatmap com os valores das frequencias ``crus'',
ou, sem nenhuma normalizacao, nao observamos as frequencias das SNPs
agrupando as sequencias de morcegos e pangolins, como seria esperado,
considerando-as sequencias do Extremo Oriente. Como verao abaixo, a
normalizacao permite que consigamos visualizar o agrupamento das
sequencias de morcegos, pangolins e de virus isolados na

\begin{Shaded}
\begin{Highlighting}[]
\KeywordTok{library}\NormalTok{(}\StringTok{"pheatmap"}\NormalTok{)}
\KeywordTok{library}\NormalTok{(}\StringTok{"RColorBrewer"}\NormalTok{)}
\CommentTok{#setwd("~/Desktop/Gepoliano/SIP2020/Code")}

\NormalTok{heatmap_table <-}\StringTok{ }\KeywordTok{read.table}\NormalTok{(}\StringTok{"~/Desktop/Gepoliano/SIP2020/Code/temp_file.txt"}\NormalTok{, }\DataTypeTok{row.names =} \DecValTok{1}\NormalTok{, }\DataTypeTok{header =} \OtherTok{TRUE}\NormalTok{, }\DataTypeTok{sep =} \StringTok{"}\CharTok{\textbackslash{}t}\StringTok{"}\NormalTok{)}
\NormalTok{heatmap_table =}\StringTok{ }\KeywordTok{as.matrix}\NormalTok{(heatmap_table)}

\CommentTok{# Escolher a cor do heatmap}
\NormalTok{col.pal <-}\StringTok{ }\KeywordTok{brewer.pal}\NormalTok{(}\DecValTok{9}\NormalTok{,}\StringTok{"Blues"}\NormalTok{)}

\CommentTok{# Definir o tipo de correlacao entre as amostras (colunas) e os genes (linhas)}
\NormalTok{drows1 <-}\StringTok{ "euclidean"}
\NormalTok{dcols1 <-}\StringTok{ "euclidean"}

\CommentTok{#Plotar o heatmap, com as diversas opcoes determinadas}
\NormalTok{hm.parameters <-}\StringTok{ }\KeywordTok{list}\NormalTok{(heatmap_table, }
                      \DataTypeTok{color =}\NormalTok{ col.pal,}
                      \DataTypeTok{cellwidth =} \DecValTok{14}\NormalTok{, }\DataTypeTok{cellheight =} \DecValTok{15}\NormalTok{, }\DataTypeTok{scale =} \StringTok{"none"}\NormalTok{,}
                      \DataTypeTok{treeheight_row =} \DecValTok{200}\NormalTok{,}
                      \DataTypeTok{kmeans_k =} \OtherTok{NA}\NormalTok{,}
                      \DataTypeTok{show_rownames =}\NormalTok{ T, }\DataTypeTok{show_colnames =}\NormalTok{ T,}
                      \CommentTok{#main = "Full heatmap (avg, eucl, unsc)",}
                      \DataTypeTok{main =} \StringTok{"Frequencies of SNP Variants of SARS-CoV-2"}\NormalTok{,}
                      \DataTypeTok{clustering_method =} \StringTok{"average"}\NormalTok{,}
                      \DataTypeTok{cluster_rows =}\NormalTok{ F, }\DataTypeTok{cluster_cols =}\NormalTok{ T,}
                      \DataTypeTok{clustering_distance_rows =}\NormalTok{ drows1, }
                      \DataTypeTok{fontsize_row =} \DecValTok{12}\NormalTok{,}
                      \DataTypeTok{fontsize_col =} \DecValTok{12}\NormalTok{,}
                      \DataTypeTok{clustering_distance_cols =}\NormalTok{ dcols1)}
\KeywordTok{do.call}\NormalTok{(}\StringTok{"pheatmap"}\NormalTok{, hm.parameters)}
\end{Highlighting}
\end{Shaded}

\includegraphics{4-6-Frequência-Alélica-em-Heatmap_files/figure-latex/heatmap1-1.pdf}

\hypertarget{plotagem-tabela-frequuxeancia-aluxe9lica-com-normalizauxe7uxe3o}{%
\subsubsection{3.2.3) Plotagem Tabela Frequência Alélica Com
Normalização}\label{plotagem-tabela-frequuxeancia-aluxe9lica-com-normalizauxe7uxe3o}}

A visualizacao do Heatmap pode ser normalizada utilisando-se a escala
logaritmica. Tirar o logaritmo da representacao da expressao genica eh
um procedimento padrao, pois ajuda a homogeneizar a variancia nas
frequencias e reduzir a dimensionalidade na variancia na visializacao do
heatmap. Devido a minha experiencia na visualizacao de expressao genica
usando Heatmaps, decidi implementar tambem a normalizacao das
frequencias alelicas dos genotipos identificados de SARS-CoV-2 neste
projeto.

\begin{Shaded}
\begin{Highlighting}[]
\KeywordTok{library}\NormalTok{(}\StringTok{"pheatmap"}\NormalTok{)}
\KeywordTok{library}\NormalTok{(}\StringTok{"RColorBrewer"}\NormalTok{)}

\NormalTok{heatmap_table <-}\StringTok{ }\KeywordTok{read.table}\NormalTok{(}\StringTok{"~/Desktop/Gepoliano/SIP2020/Code/temp_file.txt"}\NormalTok{, }\DataTypeTok{row.names =} \DecValTok{1}\NormalTok{, }\DataTypeTok{header =} \OtherTok{TRUE}\NormalTok{, }\DataTypeTok{sep =} \StringTok{"}\CharTok{\textbackslash{}t}\StringTok{"}\NormalTok{)}
\NormalTok{heatmap_table =}\StringTok{ }\KeywordTok{as.matrix}\NormalTok{(heatmap_table)}

\NormalTok{log_table_}\DecValTok{09}\NormalTok{_}\DecValTok{18}\NormalTok{_}\DecValTok{2020}\NormalTok{ =}\StringTok{ }\KeywordTok{log}\NormalTok{ (heatmap_table }\OperatorTok{+}\StringTok{ }\DecValTok{1}\NormalTok{)}

\CommentTok{# Escolher a cor do heatmap}
\NormalTok{col.pal <-}\StringTok{ }\KeywordTok{brewer.pal}\NormalTok{(}\DecValTok{9}\NormalTok{,}\StringTok{"Blues"}\NormalTok{)}

\CommentTok{# Definir o tipo de correlacao entre as amostras (colunas) e os genes (linhas)}
\NormalTok{drows1 <-}\StringTok{ "euclidean"}
\NormalTok{dcols1 <-}\StringTok{ "euclidean"}

\CommentTok{#Plotar o heatmap, com as diversas opcoes determinadas}
\NormalTok{hm.parameters <-}\StringTok{ }\KeywordTok{list}\NormalTok{(log_table_}\DecValTok{09}\NormalTok{_}\DecValTok{18}\NormalTok{_}\DecValTok{2020}\NormalTok{, }
                      \DataTypeTok{color =}\NormalTok{ col.pal,}
                      \DataTypeTok{cellwidth =} \DecValTok{14}\NormalTok{, }\DataTypeTok{cellheight =} \DecValTok{15}\NormalTok{, }\DataTypeTok{scale =} \StringTok{"none"}\NormalTok{,}
                      \DataTypeTok{treeheight_row =} \DecValTok{200}\NormalTok{,}
                      \DataTypeTok{kmeans_k =} \OtherTok{NA}\NormalTok{,}
                      \DataTypeTok{show_rownames =}\NormalTok{ T, }\DataTypeTok{show_colnames =}\NormalTok{ T,}
                      \CommentTok{#main = "Full heatmap (avg, eucl, unsc)",}
                      \DataTypeTok{main =} \StringTok{"Frequencies of SNP Variants of SARS-CoV-2"}\NormalTok{,}
                      \DataTypeTok{clustering_method =} \StringTok{"average"}\NormalTok{,}
                      \DataTypeTok{cluster_rows =}\NormalTok{ F, }\DataTypeTok{cluster_cols =}\NormalTok{ T,}
                      \DataTypeTok{clustering_distance_rows =}\NormalTok{ drows1, }
                      \DataTypeTok{fontsize_row =} \DecValTok{12}\NormalTok{,}
                      \DataTypeTok{fontsize_col =} \DecValTok{12}\NormalTok{,}
                      \DataTypeTok{clustering_distance_cols =}\NormalTok{ dcols1)}
\KeywordTok{do.call}\NormalTok{(}\StringTok{"pheatmap"}\NormalTok{, hm.parameters)}
\end{Highlighting}
\end{Shaded}

\hypertarget{visualizauxe7uxe3o-de-filogenias-usando-r}{%
\section{4) Visualização de Filogenias usando
R}\label{visualizauxe7uxe3o-de-filogenias-usando-r}}

Nesta seção, usamos a aba ``phylo'' do projeto VCF-kit, a qual inclui
instruções de como usar este pacote, para tentar visualizar as relações
de variação entre sequências de SARS-CoV-2 diretamente a partir dos
arquivos VCF. Para usar este programa, precisamos de um tipo especial de
arquivo chamado Newick. Necessitamos então, transformar arquivos VCF, em
Newick. Abaixo a plotagem é testada usando um arquivo Newick, o arquivo
treefile.newick.

\url{https://vcf-kit.readthedocs.io/en/latest/phylo/}

\hypertarget{instalauxe7uxe3o-de-pacotes}{%
\subsection{4.1) Instalação de
Pacotes}\label{instalauxe7uxe3o-de-pacotes}}

Para instalação de livrarias, o erro \emph{R version 3.5 or greater,
install Bioconductor packages using BiocManager}

aponta para a necessidade de utilização de BiocManager. Um exemplo do
comando para instalação de BiocManager é a linha presente em minhas
análises GSEA:

BiocManager::install(``clusterProfiler'', version = ``3.8'')

Com a linha abaixo, podemos determinar a versão do software que queremos
usar.

\begin{Shaded}
\begin{Highlighting}[]
\KeywordTok{install.packages}\NormalTok{(}\StringTok{"tidyverse"}\NormalTok{)}
\NormalTok{BiocManager}\OperatorTok{::}\KeywordTok{install}\NormalTok{(}\KeywordTok{c}\NormalTok{(}\StringTok{'ape'}\NormalTok{,}\StringTok{'phyloseq'}\NormalTok{,}\StringTok{'ggmap'}\NormalTok{), }\DataTypeTok{suppressUpdates =} \OtherTok{TRUE}\NormalTok{)}
\end{Highlighting}
\end{Shaded}

\hypertarget{visualizauxe7uxe3o-do-arquivo-newick-teste}{%
\subsection{4.2) Visualização do Arquivo Newick
teste}\label{visualizauxe7uxe3o-do-arquivo-newick-teste}}

Nesta parte, usamos o arquivo em formato newick. No caso abaixo, o
arquivo pode ser encontrado no seguinte link:
\url{http://etetoolkit.org/treeview/} .

\begin{Shaded}
\begin{Highlighting}[]
\KeywordTok{library}\NormalTok{(tidyverse)}
\KeywordTok{library}\NormalTok{(ape)}
\KeywordTok{library}\NormalTok{(ggmap)}
\KeywordTok{library}\NormalTok{(phyloseq)}

\NormalTok{tree <-}\StringTok{ }\NormalTok{ape}\OperatorTok{::}\KeywordTok{read.tree}\NormalTok{(}\KeywordTok{paste0}\NormalTok{(}\StringTok{"~/Desktop/Gepoliano/UFSB/vcf-kit/treefile.newick"}\NormalTok{))}

\CommentTok{# Optionally set an outgroup.}
\CommentTok{# tree <- root(tree,outgroup = "outgroup", resolve.root = T)}

\NormalTok{treeSegs <-}\StringTok{ }\NormalTok{phyloseq}\OperatorTok{::}\KeywordTok{tree_layout}\NormalTok{(}
\NormalTok{                                phyloseq}\OperatorTok{::}\KeywordTok{phy_tree}\NormalTok{(tree),}
                                \DataTypeTok{ladderize =}\NormalTok{ T}
\NormalTok{                                )}

\NormalTok{treeSegs}\OperatorTok{$}\NormalTok{edgeDT <-}\StringTok{ }\NormalTok{treeSegs}\OperatorTok{$}\NormalTok{edgeDT  }\OperatorTok\StringTok{ }
\StringTok{                   }\NormalTok{dplyr}\OperatorTok{::}\KeywordTok{mutate}\NormalTok{(}\DataTypeTok{edge.length =} 
                                    \KeywordTok{ifelse}\NormalTok{(edge.length }\OperatorTok{<}\StringTok{ }\DecValTok{0}\NormalTok{, }\DecValTok{0}\NormalTok{, edge.length)}
\NormalTok{                                 , }\DataTypeTok{xright =}\NormalTok{ xleft }\OperatorTok{+}\StringTok{ }\NormalTok{edge.length}
\NormalTok{                                 )}
\NormalTok{edgeMap =}\StringTok{ }\KeywordTok{aes}\NormalTok{(}\DataTypeTok{x =}\NormalTok{ xleft, }\DataTypeTok{xend =}\NormalTok{ xright, }\DataTypeTok{y =}\NormalTok{ y, }\DataTypeTok{yend =}\NormalTok{ y)}
\NormalTok{vertMap =}\StringTok{ }\KeywordTok{aes}\NormalTok{(}\DataTypeTok{x =}\NormalTok{ x, }\DataTypeTok{xend =}\NormalTok{ x, }\DataTypeTok{y =}\NormalTok{ vmin, }\DataTypeTok{yend =}\NormalTok{ vmax)}
\NormalTok{labelMap <-}\StringTok{ }\KeywordTok{aes}\NormalTok{(}\DataTypeTok{x =}\NormalTok{ xright}\FloatTok{+0.0001}\NormalTok{, }\DataTypeTok{y =}\NormalTok{ y, }\DataTypeTok{label =}\NormalTok{ OTU)}

\KeywordTok{ggplot}\NormalTok{(}\DataTypeTok{data =}\NormalTok{ treeSegs}\OperatorTok{$}\NormalTok{edgeDT) }\OperatorTok{+}\StringTok{ }\KeywordTok{geom_segment}\NormalTok{(edgeMap) }\OperatorTok{+}\StringTok{ }
\StringTok{  }\KeywordTok{geom_segment}\NormalTok{(vertMap, }\DataTypeTok{data =}\NormalTok{ treeSegs}\OperatorTok{$}\NormalTok{vertDT) }\OperatorTok{+}
\StringTok{  }\KeywordTok{geom_text}\NormalTok{(labelMap, }\DataTypeTok{data =}\NormalTok{ dplyr}\OperatorTok{::}\KeywordTok{filter}\NormalTok{(treeSegs}\OperatorTok{$}\NormalTok{edgeDT, }\OperatorTok{!}\KeywordTok{is.na}\NormalTok{(OTU)), }\DataTypeTok{na.rm =} \OtherTok{TRUE}\NormalTok{, }\DataTypeTok{hjust =} \FloatTok{-0.05}\NormalTok{) }\OperatorTok{+}
\StringTok{  }\NormalTok{ggmap}\OperatorTok{::}\KeywordTok{theme_nothing}\NormalTok{() }\OperatorTok{+}\StringTok{ }
\StringTok{  }\KeywordTok{scale_x_continuous}\NormalTok{(}\DataTypeTok{limits =} \KeywordTok{c}\NormalTok{(}
    \KeywordTok{min}\NormalTok{(treeSegs}\OperatorTok{$}\NormalTok{edgeDT}\OperatorTok{$}\NormalTok{xleft)}\OperatorTok{-}\FloatTok{0.15}\NormalTok{,}
    \KeywordTok{max}\NormalTok{(treeSegs}\OperatorTok{$}\NormalTok{edgeDT}\OperatorTok{$}\NormalTok{xright)}\OperatorTok{+}\FloatTok{0.15}
\NormalTok{  ),}
  \DataTypeTok{expand =} \KeywordTok{c}\NormalTok{(}\DecValTok{0}\NormalTok{,}\DecValTok{0}\NormalTok{))}
\end{Highlighting}
\end{Shaded}

\end{document}
